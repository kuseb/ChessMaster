 \documentclass[12pt]{article}
\usepackage{polski}
\usepackage[utf8]{inputenc}
\usepackage{graphicx}
\usepackage{listings}
\usepackage{color}

\definecolor{dkgreen}{rgb}{0,0.6,0}
\definecolor{gray}{rgb}{0.5,0.5,0.5}
\definecolor{mauve}{rgb}{0.58,0,0.82}

\lstset{frame=tb,
  aboveskip=3mm,
  belowskip=3mm,
  showstringspaces=false,
  columns=flexible,
  basicstyle={\small\ttfamily},
  numbers=none,
  numberstyle=\tiny\color{gray},
  keywordstyle=\color{blue},
  commentstyle=\color{dkgreen},
  stringstyle=\color{mauve},
  breaklines=true,
  breakatwhitespace=true,
  tabsize=3
}


\begin{document}

\begin{titlepage}
\centering
\includegraphics[width=0.15\textwidth]{logo}\par\vspace{1cm}
{\scshape\LARGE Politechnika Warszawska \par}
\vspace{1cm}
{\huge\bfseries  Raport z testów aplikacji \\ Chess Master \linebreak \\  \par}
\vspace{1cm}
{\bfseries Projekt Indywidualny \par}
\vspace{2cm}
{\Large\itshape  Sebastian Kurpios \par}
\end{titlepage}

\tableofcontents
\newpage

\section{Tester}
Z powodu opóźnienia w realizacji projektu testerem został jej twórca: Sebastian Kurpios, który postarał się rzetelnie wykonać to zadanie.

\section{Platformy}
Zdecydowałem się przetestować aplikację na 2 platformach: laptopie oraz komputerze stacjonarnym. \\
Specyfikacja sprzętowa:
\begin{itemize}
\item Laptop
\begin{itemize}
\item Producent: MSI
\item Model: GE40 20C Dragon Eyes
\item Procesor: Intel® Core™ i7 4-tej generacji
\item Pamięć RAM: 16 GB
\item Karta Graficzna: Nvidia GeForce GTX 760M 
\item System operacyjny: Windows 10
\end{itemize}
\item Komputer stacjonarny
\begin{itemize}
\item Producent: Vobis
\item Model: W7P E6300
\item Procesor:  Intel Pentium Dual Core E6300
\item Pamięć RAM: 2 GB
\item Karta Graficzna: ATI RADEON HD 4670
\item System operacyjny: Windows 7
\end{itemize}
\end{itemize} 
\newpage
\section{Testy}
\subsection{Testy modelu danych i funkcjonalności}
\begin{enumerate}
\item Test poprawnego wyznaczania ruchów Laptop: OK, Komputer stacjonarny: OK
\item Test poprawnego aktualizowania danych: Laptop: OK, Komputer stacjonarny: OK
\item Test usuwania z modelu zbitych figur: Laptop: OK, Komputer stacjonarny: OK
\item Test wczytywania planszy (pliku .3ds): Laptop: OK, Komputer stacjonarny: OK
\item Test wczytywania figur: Laptop: OK, Komputer stacjonarny: OK
\item Test podmiany siatki figury (promocja piona): Laptop: OK, Komputer stacjonarny: OK
\item Test logowania: Laptop: BŁĄD, Komputer stacjonarny: BŁĄD - UWAGA! Brak implementacji z powodu braku serwera
\item Test rejestracji: Laptop: BŁĄD, Komputer stacjonarny: BŁĄD - UWAGA! Brak implementacji z powodu braku serwera
\end{enumerate}
\subsection{Testy interfejsu}
\subsubsection{Testy interfejsu planszy}
\begin{enumerate}
\item Test naciśnięcia na wszystkie pionki - Laptop: OK, Komputer stacjonarny: OK
\item Test wybrania innego pola niż zaznaczone: Laptop: OK, Komputer stacjonarny: OK
\item Test wybrania właściwego pola: Laptop: OK, Komputer stacjonarny: OK
\item Test wykonania się animacji dla wszystkich figur: Laptop: OK, Komputer stacjonarny: OK
\item Test zbijania i usuwania z pola figur: Laptop: OK, Komputer stacjonarny: OK
\item Test roszady i bicia w przelocie:
Laptop: OK, Komputer stacjonarny: OK
\item Test końca czasu:
Laptop: OK, Komputer stacjonarny: OK
\item Test poprawności wyświetlania wykonanych ruchów
Laptop: OK, Komputer stacjonarny: OK
\item Test czatu: Laptop: OK, Komputer stacjonarny: OK - UWAGA! Wiadomości wyświetlają się, ale nie są nigdzie wysyłane z powodu braku serwera!
\end{enumerate}
\subsubsection{Testy interfejsu logowania i rejestracji}
\begin{enumerate}
\item Test zalogowania się - Laptop: OK, Komputer stacjonarny: OK - UWAGA! można zalogować się używając dowolnego hasła i loginu, ponieważ nie ma serwera, po zalogowaniu zmienia się zdjęcie profilowe oraz nazwa użytkownika
\item Test rejestracji - tak samo jak w przypadku logowania, dane nie zapisują się nigdzie!
\item Test logowania przez \textit{Facebook} - Laptop: BRAK REAKCJI, Komputer stacjonarny: BRAK REAKCJI - funkcjonalność zostanie wykonana w przyszłości
\end{enumerate}
\subsection{Testy wydajnościowe}
\begin{enumerate}
\item Laptop - Aplikacja działa płynnie, figury jak i animacje poruszają się zgodnie z oczekiwaniami użytkownika, "przycięcia" aplikacji nie występują
\item Komputer stacjonarny - Aplikacja działa płynnie, chociaż czasami można zaobserwować "przycinanie się" aplikacji. Jest to spowodowane najprawdopodobniej przez słabszą architekturę tej jednostki, jak i również może niewydajną i nieoptymalną implementacje
\end{enumerate}
\section{Interfejs}
Interfejs jest czytelny i intuicyjny. Grafika 3D robi wrażenie, chociaż kolor tła oraz czcionka, która została użyta, mogą nie przypaść wszystkim do gustu.
\section{Jakość kodu}
Kod zawiera dosyć mało komentarzy i na pierwszy rzut oka wydaje się dosyć nieczytelny. Nazwy zmiennych i klas są dosyć intuicyjne, ale niektóre z metod są za długie i wykonują więcej niż atomowa funkcjonalność, jaką powinny.

\section{Podumowanie}
Aplikacja posiadała kilka błędów, ale są one spowodowane głównie przez brak implementacji dancyh modułów na tym etapie rozwoju. Interfejs jest intuicyjny i względnie ładny. Niestety, jakość kodu zostawia trochę do życzenia i powinna zostać poprawiona. 
\end{document}